\appendix
\section{Information for Windows Users}\label{app:windows}
If you are using HeidelTime standalone on Windows, you have to download and install a Perl interpreter, e.g. ActivePerl from \url{http://www.activestate.com/activeperl}, as well as the Windows version of the TreeTagger~\cite{Schmid1994}, including parameter files for the languages you want to process. A set of initial files to download and extract to the same folder are the following (newer versions may be available):
\begin{itemize}
  \item The Windows Version of the TreeTagger: \\
    \url{http://www.cis.uni-muenchen.de/~schmid/tools/TreeTagger/data/tree-tagger-windows-3.2.zip}
  \item The tagging scripts: \\
    \url{http://www.cis.uni-muenchen.de/~schmid/tools/TreeTagger/data/tagger-scripts.tar.gz}
\end{itemize}

As for the parameter files for the respective languages, you will need to put any \texttt{.par} files in the \texttt{lib/} folder, any \texttt{\textit{language}-abbreviations} in \texttt{lib/} and any \texttt{tree-tagger-\textit{language}} script file in \texttt{cmd/}. \\

Once this is set up, you will need to specify the \textit{treeTaggerHome}-Variable in \texttt{config.props} as described in Section~\ref{sec:Installation_Configuration}. After that, you should be able to run \product{} as described in Section~\ref{sec:Usage_CommandLine}.