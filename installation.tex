\section{Installation}\label{sec:Installation}
This section explains the steps necessary to use \product{}.

\subsection{Files}\label{sec:Files}
\product{} comes with three files and two folders:
\begin{itemize}
\itemWithHead{\textnormal{\executableFile}}
Executable java file; see Section~\ref{sec:Usage} for more information about possible command line arguments.
\itemWithHead{\textnormal{\configFile}}
Configuration file; it has to be located in the same directory as the executable. See Section~\ref{sec:Installation_Configuration} for more information about the configuration of \product{}.
\itemWithHead{\textnormal{\textbf{src/}}}
Folder containing the source files that were used to generate the executable jar file \executableFile.
\itemWithHead{\textnormal{\textbf{doc/}}}
Folder containing the Javadoc files.
\itemWithHead{\textnormal{\manualFile}}
This file.

\end{itemize}

\subsection{Prerequisites}\label{sec:Installation_Prerequisites}
\product{} requires the following two components to be installed:
\begin{enumerate}
\item The Java Runtime Environment~\cite{Java} and
\item A compatible pre-processing tagger that is capable of identifying language tokens, part of speech and sentence boundaries in all languages supported by HeidelTime. We decided to use TreeTagger~\cite{TreeTagger} for English, German, Dutch, Spanish, Italian, Russian, French and Chinese. You will need to download and install so called "parameter files" for those languages as well (all that are available, e.g., for German, download the Latin1 and the UTF-8 variants), to provide TreeTagger with the necessary functionality (see the TreeTagger website for more information).
\item To process Chinese documents, please grab a copy of the Chinese TreeTagger parameter file from Serge Sharoff's page \url{http://corpus.leeds.ac.uk/tools/zh/} as well as a copy of the Chinese Tokenizer \url{https://drive.google.com/uc?id=0BwqFBQjz9NUiZ3kybkc4YTliMzA}. Extract the parameter files into the TreeTagger home directory so the files from the \texttt{lib} and \texttt{cmd} folders land in the TreeTager folders. Extract the tokenizer into its own directory and remember the path for the configuration later (Section \ref{sec:Installation_Configuration}).
\item To process Russian documents, please grab a copy of the Russian parameter file by Serge Sharoff from \url{http://corpus.leeds.ac.uk/mocky/} and extract it into the TreeTagger's \texttt{lib} folder.
\item If you use \product{} to annotate documents in Vietnamese, you will need to get a copy of JVnTextPro~\cite{JVnTextPro}
\item For Arabic documents, you will need to download a full package of the Stanford POS Tagger~\cite{StanfordPOSTagger}
\item In order to process documents in Croatian, you will need to download a copy of hunpos~\cite{hunpos} as well as the Croatian tagger model file for it from \url{http://nlp.ffzg.hr/resources/models/tagging/}.

Note 2: If you use \product{} on Windows, please see Appendix \ref{app:windows}.
\end{enumerate}

\subsection{Configuration}\label{sec:Installation_Configuration}
After the installation of the prerequisites mentioned in Section~\ref{sec:Installation_Prerequisites}, there are a few parameters to set up in the configuration file \configFile{}:
\begin{itemize}

\subsection*{\textbf{For most languages}}
\itemWithHead{\textnormal{\configVar{treeTaggerHome}}}
This variable has to point to the root directory of TreeTagger that you will need to use for most languages.
Example: \texttt{/opt/treetagger/}

\subsection*{\textbf{For Chinese}}
\itemWithHead{\textnormal{\configVar{chineseTokenizerPath}}}
This variable has to point to the directory where the Chinese Tokenizer Script and files are.
Example: \texttt{/opt/treetagger/chinese-tokenizer/}

\subsection*{\textbf{For use with Vietnamese}}
\itemWithHead{\textnormal{\configVar{word\_model\_path}}}
This variable needs to point to the \textit{folder} where JVnTextPro's segmentation model is stored. \\
Example: \texttt{/opt/jvntextpro/models/jvnsegmenter}
\itemWithHead{\textnormal{\configVar{sent\_model\_path}}}
This variable needs to point to the \textit{folder} where JVnTextPro's sentence segmentation model is stored. \\
Example: \texttt{/opt/jvntextpro/models/jvnsensegmenter}
\itemWithHead{\textnormal{\configVar{pos\_model\_path}}}
This variable needs to point to the \textit{folder} where JVnTextPro's part of speech model is stored. \\
Example: \texttt{/opt/jvntextpro/models/jvnpostag/maxent}

\subsection*{\textbf{For use with Arabic}}
\itemWithHead{\textnormal{\configVar{model\_path}}}
This variable needs to point to the path where StanfordPOSTagger's tagger model \textit{file} is stored. \\
Example: \texttt{/opt/stanfordpostagger/models/arabic.tagger}
\itemWithHead{\textnormal{\configVar{config\_path}}}
This variable can be set to point to the path where StanfordPOSTagger's config model \textit{file} is stored. This setting is optional and can be left empty. \\
Example: \texttt{/opt/stanfordpostagger/tagger.config}

\subsection*{\textbf{For use with Croatian}}
\itemWithHead{\textnormal{\configVar{hunpos\_path}}}
This variable must point to the \textbf{folder} where the hunpos executable is located. \\
Example: \texttt{/opt/hunpos/}
\itemWithHead{\textnormal{\configVar{hunpos\_model\_path}}}
This variable needs to represent the \textbf{name} of the hunpos model file used, residing in the \texttt{hunpos\_path} set above. \\
Example: \texttt{model.hunpos.mte5.defnpout}

\subsection*{\textbf{General options}}
\itemWithHead{\textnormal{\configVar{considerDate}}}
Indicates whether HeidelTime should consider Timex3 expressions of type DATE.
\itemWithHead{\textnormal{\configVar{considerDuration}}}
Indicates whether HeidelTime should consider Timex3 expressions of type DURATION.
\itemWithHead{\textnormal{\configVar{considerSet}}}
Indicates whether HeidelTime should consider Timex3 expressions of type SET.
\itemWithHead{\textnormal{\configVar{considerTime}}}
Indicates whether HeidelTime should consider Timex3 expressions of type TIME.
\end{itemize}
All other options are not meant to be changed and therefore skipped in this section.
