\documentclass[
     11pt,         % font size
     a4paper,      % paper format
     oneside,
     ]{article}
\usepackage{algorithm}
\usepackage{algorithmic}
\usepackage{color}
\usepackage{enumitem}
\usepackage{listings}
\usepackage{url}
%\usepackage{times}

\usepackage{ifpdf}
\usepackage{ifthen}
\usepackage{graphicx}

\usepackage{longtable}
\usepackage{ragged2e}
\usepackage{array}

\usepackage[T1]{fontenc}
%\usepackage[utf8]{inputenc}
\usepackage[USenglish]{babel}
\usepackage[numbers]{natbib}

%\ifpdf
\usepackage[pdftex, bookmarks=true]{hyperref}
%\else
%\fi

\newcommand{\commandLineCmd}[1]{"#1"}

% Formats config variable
\newcommand{\configVar}[1]{\emph{#1}}

% Formats file name
\newcommand{\file}[1]{\texttt{#1}}
\newcommand{\configFile}{\file{config.props}}
\newcommand{\executableFile}{\file{\executableFilePlain}}
\newcommand{\executableFilePlain}{de.unihd.dbs.heideltime.standalone.jar}
\newcommand{\manualFile}{\file{Manual.pdf}}

% Sets up head for item in itemize or enumerate
\newcommand{\itemWithHead}[1]{\item \emph{#1}\\}

% Sets up head for item in itemize or enumerate with no followed newline
\newcommand{\itemWithHeadNoNL}[1]{\item \emph{#1}}

% Formats java expression
\newcommand{\javaExpr}[1]{\emph{#1}}

% Comments sth.
\newcommand{\note}[1]{\textcolor{red}{\textbf{#1}}}

% Product name
\newcommand{\product}{HeidelTime Standalone}

% Strongly emphasizes
\newcommand{\strong}[1]{\textbf{#1}}

% Formats table head
\newcommand{\tableHead}[1]{\textsc{#1}}

% 
\newcommand{\languages}{English, German, Spanish, Italian, Vietnamese, Arabic, and Dutch}


% Document
\begin{document}

% Title
\title{%
HeidelTime Standalone Version\\
Manual
}
\author{Julian Zell, Andreas Fay, Jannik Str\"otgen (Heidelberg University)\\[0.2em]
\small \texttt{zell@informatik.uni-heidelberg.de, stroetgen@uni-hd.de}
}
\date{March 2014}
\maketitle

\begin{abstract}
This document contains information about how to install and use the standalone version of HeidelTime. HeidelTime itself is a multilingual temporal tagger for the extraction and normalization of temporal expressions from documents, developed at the Heidelberg University from Str\"otgen and Gertz~\cite{Strotgen2010, HeidelTime, StroetgenGertz2012}.\\
The original version of HeidelTime is designed to run within a proper UIMA-Pipeline~\cite{UIMA}. With this standalone version the original version is wrapped such that it can be run with less prerequisites and especially without UIMA.\\
HeidelTime Standalone comes with resources for \languages. Dutch resources were developed and kindly provided by Matje van de Camp (Tilburg University)\cite{Matje}. French resources were provided by Véronique Moriceau (LIMSI-CNRS)\cite{Veronique}.
\end{abstract}

% Table of contents
\tableofcontents

% Chapters
\section{Preface}\label{sec:Preface}
This document contains information about how to install and use the standalone version of HeidelTime. HeidelTime itself is a multilingual temporal tagger for the extraction and normalization of temporal expressions from documents, developed at the University of Heidelberg from Str\"otgen and Gertz~\cite{Strotgen2010, HeidelTime, StroetgenGertz2013}\\
The original version of HeidelTime is designed to run within a proper UIMA-Pipeline~\cite{UIMA}. With this standalone version the original version is wrapped such that it can be run with less prerequisites and especially without UIMA.

\section{Quick Start}\label{sec:QuickStart}
This section will briefly outline what is necessary in order to get \product{} going. See Section~\ref{sec:Installation} for a more detailed description.
\begin{enumerate}
\item Install Java Runtime Environment~\cite{Java} in order execute Java programs.
\item Install TreeTagger~\cite{TreeTagger} of the University of Stuttgart with the parameter files for English, German, Dutch, Spanish, Italian, French, Chinese and Russian.
\item Ensure the path to your local TreeTagger installation is set correctly. Therefore, check the variable \configVar{treeTaggerHome} in \configFile{}. It has to point to the root directory of your TreeTagger installation.
\item Change to the directory containing \executableFile{}.
\item Run \product{} using \newline \commandLineCmd{java -jar \executableFilePlain{} <file>} \newline where \emph{<file>} is the path to a text document.
\end{enumerate}

\section{Installation}\label{sec:Installation}
This section explains the steps necessary to use \product{}.

\subsection{Files}\label{sec:Files}
\product{} comes with three files and two folders:
\begin{itemize}
\itemWithHead{\textnormal{\executableFile}}
Executable java file; see Section~\ref{sec:Usage} for more information about possible command line arguments.
\itemWithHead{\textnormal{\configFile}}
Configuration file; it has to be located in the same directory as the executable. See Section~\ref{sec:Installation_Configuration} for more information about the configuration of \product{}.
\itemWithHead{\textnormal{\textbf{src/}}}
Folder containing the source files that were used to generate the executable jar file \executableFile.
\itemWithHead{\textnormal{\textbf{doc/}}}
Folder containing the Javadoc files.
\itemWithHead{\textnormal{\manualFile}}
This file.

\end{itemize}

\subsection{Prerequisites}\label{sec:Installation_Prerequisites}
\product{} requires the following two components to be installed:
\begin{enumerate}
\item The Java Runtime Environment~\cite{Java} and
\item A compatible pre-processing tagger that is capable of identifying language tokens, part of speech and sentence boundaries in all languages supported by HeidelTime. We decided to use TreeTagger~\cite{TreeTagger} of the University of Stuttgart for English, German, Dutch, Spanish and Italian. You will need to download and install the so called "parameter files" for those languages as well (all that are available, e.g., for German, download the Latin1 and the UTF-8 parameter files), to provide TreeTagger with the necessary functionality (see the TreeTagger Web-site for more information).
\item If you use \product{} to annotate documents in Vietnamese, you will need to get a copy of JVnTextPro~\cite{JVnTextPro}
\item For Arabic documents, you will need to download a full package of the Stanford POS Tagger~\cite{StanfordPOSTagger}

Note 2: If you use \product{} on Windows, please see Appendix \ref{app:windows}.
\end{enumerate}

\subsection{Configuration}\label{sec:Installation_Configuration}
After the installation of the prerequisites mentioned in Section~\ref{sec:Installation_Prerequisites}, there are a few parameters to set up in the configuration file \configFile{}:
\begin{itemize}

\subsection*{\textbf{For most languages}}
\itemWithHead{\textnormal{\configVar{treeTaggerHome}}}
This variable has to point to the root directory of TreeTagger that you will need to use for most languages.

\subsection*{\textbf{For use with Vietnamese}}
\itemWithHead{\textnormal{\configVar{word\_model\_path}}}
This variable needs to point to the \textit{folder} where JVnTextPro's segmentation model is stored. \\
Example: \texttt{/opt/jvntextpro/models/jvnsegmenter}
\itemWithHead{\textnormal{\configVar{sent\_model\_path}}}
This variable needs to point to the \textit{folder} where JVnTextPro's sentence segmentation model is stored. \\
Example: \texttt{/opt/jvntextpro/models/jvnsensegmenter}
\itemWithHead{\textnormal{\configVar{pos\_model\_path}}}
This variable needs to point to the \textit{folder} where JVnTextPro's part of speech model is stored. \\
Example: \texttt{/opt/jvntextpro/models/jvnpostag/maxent}

\subsection*{\textbf{For use with Arabic}}
\itemWithHead{\textnormal{\configVar{model\_path}}}
This variable needs to point to the path where StanfordPOSTagger's tagger model \textit{file} is stored. \\
Example: \texttt{/opt/stanfordpostagger/models/arabic-accurate.tagger}
\itemWithHead{\textnormal{\configVar{config\_path}}}
This variable can be set to point to the path where StanfordPOSTagger's config model \textit{file} is stored. This setting is optional and can be omitted (left empty). \\
Example: \texttt{/opt/stanfordpostagger/tagger.config}

\subsection*{\textbf{General options}}
\itemWithHead{\textnormal{\configVar{considerDate}}}
Indicates whether HeidelTime should consider Timex3 expressions of type DATE.
\itemWithHead{\textnormal{\configVar{considerDuration}}}
Indicates whether HeidelTime should consider Timex3 expressions of type DURATION.
\itemWithHead{\textnormal{\configVar{considerSet}}}
Indicates whether HeidelTime should consider Timex3 expressions of type SET.
\itemWithHead{\textnormal{\configVar{considerTime}}}
Indicates whether HeidelTime should consider Timex3 expressions of type TIME.
\end{itemize}
All other options are not meant to be changed and therefore skipped in this section.

\section{Usage}\label{sec:Usage}
This section explains how to use \product{} both as a command line tool and as a component in other Java projects.

\subsection{Command Line Usage}\label{sec:Usage_CommandLine}
To use \product{}, open a command line terminal and switch to the directory containing \executableFile{}. You then are able to run it using the following command:\newline \commandLineCmd{java -jar \executableFilePlain{} <file> [options]} where \emph{<file>} is the path to a text document on your hard disk and \emph{[options]} are possible options explained in Table~\ref{tab:Usage_Options}.
\subsection*{Extra steps for Arabic and Vietnamese tagging}\label{taggersetup}
To tag Arabic and Vietnamese documents, you will need to utilize a different command line scheme. First, you will have to set the \texttt{HT\_CP} variable to include \product{}'s class files as well as those of the languages' respective taggers:\newline
\newline
Under Unix/Linux/Mac OS X: \newline
\indent \commandLineCmd{export HT\_CP="<\$1>:<\$2>:<\$3>:\$CLASSPATH"} \newline\newline
or under Windows: \newline
\indent \commandLineCmd{set HT\_CP=<\$1>;<\$2>;<\$3>;\%CLASSPATH\%} \newline\newline
where \newline
<\$1> is the path to JVnTextPro's \texttt{bin} folder, e.g. \texttt{/opt/jvntextpro/bin/},\newline <\$2> is the path to StanfordPOSTagger's \texttt{.jar} file, e.g. \newline\texttt{/opt/stanfordpostagger/stanford-postagger.jar} and \newline
<\$3> is \texttt{de.unihd.dbs.heideltime.standalone.jar} \newline\newline
Once you have this variable set, you can use the following command line:\newline
\texttt{java -cp \$HT\_CP de.unihd.dbs.heideltime.standalone.HeidelTimeStandalone \newline <file> [options]}\newline
where \emph{<file>} is the path to a text document on your hard disk and \emph{[options]} are possible options explained in Table~\ref{tab:Usage_Options}.


\begin{longtable}{|l|>{\RaggedRight}p{3cm}|>{\RaggedRight}p{7cm}|}
\caption{Command line arguments of \product{}.}\label{tab:Usage_Options}\\
\hline
\tableHead{Option} & \tableHead{Name} & \tableHead{Description} \\
\hline\hline\endfirsthead
\hline
\tableHead{Option} & \tableHead{Name} & \tableHead{Description} \\
\hline\hline\endhead
-dct & Document Creation Time & Date of the format YYYY-MM-DD when the document specified by \emph{<file>} was created. This information is used only if "-t" is set to NEWS or COLLOQUIAL. It is used to resolve relative temporal expression such as "today". The default value is the current date on the local machine.\\\hline
-l & Language & Language of the document. Possible values are: ENGLISH, GERMAN, DUTCH, ENGLISHCOLL (for -t COLLOQUIAL), ENGLISHSCI (for -t SCIENTIFIC), SPANISH, ITALIAN, ARABIC, VIETNAMESE, FRENCH, CHINESE, RUSSIAN, CROATIAN. The default is ENGLISH. \\\hline
-t & Type & Type of the document specified by \emph{<file>}. Possible values are: NARRATIVES, NEWS, COLLOQUIAL and SCIENTIFIC. The default value is NARRATIVES. The major difference between these types is the consideration of "-dct" if type is set to NEWS or COLLOQUIAL. \\\hline
-o & Output Type & Type of the result. Possible values are: XMI and TIMEML. The default value is TIMEML. \\\hline
-e & Encoding & Encoding of the document that is to be processed, e.g., UTF-8, ISO-8859-1, \ldots\  Default value is UTF-8.\\\hline
-c & Configuration file & Relative or absolute path to the configuration file. Default file is config.props \\\hline
-v/-vv & Verbosity & Turns on verbose or very verbose logging. \\\hline
-it & IntervalTagger & Enables the IntervalTagger and outputs recognized intervals. \\\hline
-locale & Locale & Lets you set a custom locale to run HeidelTime under. Format is: X\_Y, where X is from ISO 639 and Y is from ISO 3166, e.g.: "en\_GB" \\\hline
-pos & POS Tagger & Lets you choose a specific part of speech tagger; either STANFORDPOSTAGGER or TREETAGGER. Note that for Arabic or Vietnamese documents, we allow only StanfordPOSTagger and JVnTextPro respectively. Please take note of the prerequisites in Section~\ref{taggersetup}.\\\hline
-h & Help & Shows you a list of commands and usage information \\\hline
\end{longtable}

You may omit any of the options since they are optional. \product{} will however force you to enter a valid document path. It will output an XMI- or TimeML-document to the standard output stream containing all annotations made by HeidelTime. You may save the output to a file by using the following command:\newline \texttt{\commandLineCmd{java -jar \executableFilePlain{} <file> [options] > <outputfile>}} where \emph{<outputfile>} is the path to the document where the output will be saved into.

\textbf{Encoding settings:}
\product{} can process files of different encodings. However, independent of the input encoding, the output is always encoded as UTF-8. If the default encoding of your Java Virtual Machine is not UTF-8, \textbf{you have to set the encoding to UTF-8} using the -Dfile.encoding option:
% assumes that the encoding of the document that is to be processed is encoded in the default encoding of the Java Virtual Machine. However, you are able to set the encoding of the input file using the -Dfile.encoding option: 
\newline \texttt{\commandLineCmd{java -Dfile.encoding=UTF-8 -jar \executableFilePlain{} <file> [options]}}\newline
If the encoding of the document that is to be processed is not UTF-8, you can specify the encoding with parameter ``-e'' as described in Table~\ref{tab:Usage_Options}.

\subsection{Component in other Projects}\label{sec:Usage_Component}
To use \product{} as a component in other projects, you have to prepare the executable jar file \executableFile{}: Add the configuration file \configFile{} to the main directory of the executable using a proper archive tool. Once this is done you can copy the executable wherever you want and use it like a library. To run \product{}, instantiate an object of \javaExpr{HeidelTimeStandalone}. To do so, you simply have to provide the desired language and type that is to be processed (see Table~\ref{tab:Usage_Options} for further information). To actually run HeidelTime, you have to call \javaExpr{process} on the recently instantiated object of type \javaExpr{HeidelTimeStandalone} with the text to be processed. If this text is of type NEWS (remember your decision when instantiating a \javaExpr{HeidelTimeStandalone} object), you have to provide the document creation time as well. As a result you will get a string containing the TimeML document with all annotations made by HeidelTime for further treatment.


\section{License}
Copyright (c) 2014, Database Research Group, Institute of Computer Science, University of Heidelberg. 
All rights reserved. This program and the accompanying materials 
are made available under the terms of the GNU General Public License.

If you use HeidelTime, please cite one of the papers describing HeidelTime: \cite{Strotgen2010, StroetgenGertz2012}. Thank you.

% HeidelTime is a multilingual, cross-domain temporal tagger.
For details, see \url{http://dbs.ifi.uni-heidelberg.de/heideltime/} or \\
\url{https://code.google.com/p/heideltime/}.



% Bibliography
\bibliographystyle{plainnat}
\bibliography{literature}
\addcontentsline{toc}{section}{References}

\appendix
\section{Information for Windows Users}\label{app:windows}
If you are using HeidelTime standalone on Windows, you have to download and install the following TreeTagger~\cite{TreeTagger} scripts and resources, in addition to the Windows version of the TreeTagger:
\begin{itemize}
 \item utf8-tokenize.perl
 \item german-abbreviations-utf8
 \item dutch-abbreviations
\end{itemize}

For this, download and extract the TreeTagger tagging scripts\\
 \url{ftp://ftp.ims.uni-stuttgart.de/pub/corpora/tagger-scripts.tar.gz} (be aware that this file should be extracted in an empty folder). Then, copy the following files to your TreeTagger folders:
\begin{itemize}
 \item copy cmd/utf8-tokenize.perl to TREETAGGER\_HOME/cmd/
 \item copy lib/german-abbreviations-utf8 to TREETAGGER\_HOME/lib/
 \item copy lib/dutch-abbreviations to TREETAGGER\_HOME/lib/
\end{itemize}

You should now be able to run HeidelTime standalone on a Windows machine.


\end{document}
